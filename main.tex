\documentclass[10pt,a4paper]{article}
\usepackage[utf8]{inputenc}
\usepackage[frenchb]{babel}
\usepackage{amsmath}
\usepackage{amsfonts}
\usepackage{amssymb}
\usepackage{hyperref}
\author{Bertrand}
\title{Electromagnetique et Antennes}

\begin{document} %%% BEGIN DOC

\maketitle
\tableofcontents


%%%%%%%%%%%%%%%%%%%%%%%%%%%%%%%%%%%%%%%%%%%%%%%%%%%%%%%%%%%%%%%%%%%%%%%%%
%%%%%%%%%%%%%%%%%%%%%%%%%%%%% RAPPELS MATHEMATIQUES %%%%%%%%%%%%%%%%%%%%%%%%%%%%%
%%%%%%%%%%%%%%%%%%%%%%%%%%%%%%%%%%%%%%%%%%%%%%%%%%%%%%%%%%%%%%%%%%%%%%%%%
\section{Rappels Mathématiques}

%%%%%%%%%%%%% DIVERGENCE %%%%%%%%%%%%%
\subsection{Divergence \cite{divergence}}
\subsubsection{Définition}
La divergence d'un champ de vecteur mesure le défaut de conservation du volume sous l'action du flot de ce champ. Le théorème de flux-divergence \cite{fluxdiv} précise que l'intégrale de la divergence d'un champ de vecteurs sur un volume V définit par une surface fermée S, est égal au flux de ce champ de vecteurs à travers la surface S. Le flux est représenté par un champ de vecteurs et la divergence de ce flux de vecteur représente en chaque point la force de la source ou de la perte en ce point. De manière simplifiée : la somme de tout les gains moins la somme de toutes les pertes est égale au flux. 
\begin{equation}
\iiint_{V}div(\overrightarrow{F})\cdot dV = \iint_{S}\overrightarrow{F}\cdot \overrightarrow{dS}
\end{equation}

$div(\overrightarrow{X})$ est une fonction à valeurs réelles qui mesure la variation première du volume le long des trajectoire dudit champ.
En dimension 3 et en coordonnées cartésiennes pour un champ de vecteurs $\overrightarrow{F}$:
\begin{equation}
div(\overrightarrow{F}) = \overrightarrow{\nabla}\cdot\overrightarrow{F}= \frac{\partial F_{x}}{\partial x} + \frac{\partial F_{y}}{\partial y} + \frac{\partial F_{z}}{\partial z}
\end{equation}

Un champ à divergence nulle est un champ qui conserve le volume, tel que le champ des vecteurs vitesse d'un fluide incompressible.

De manière générale, en physique la divergence est reliée à l'expression locale de la propriété de conservation d'une grandeur. En considérant une surface quelconque S, la variation d'une grandeur conservative dans le volume fermé par cette surface est, par définition une grandeur conservative car il n'existe pas de source de création ou de destruction d'une grandeur conservative. Le bilan de cette grandeur entre deux instants est donc uniquement égale à la somme du flux de cette grandeur à travers la surface S + la variation temporelle de la grandeur à l'intérieur de la surface S. Si cette grandeur est conservative alors le bilan est nul. 

D'où en électromagnétisme par exemple, avec $\overrightarrow{J}$ le vecteur densité de courant et $\rho$ la densité de charges:
\begin{equation}
\iint_{S}\overrightarrow{J}\cdot \overrightarrow{dS} + \frac{d}{dt}\iiint_{V}\rho\cdot dV = 0
\end{equation}


\subsubsection{Propriétés}
Un champ rotationnel est à divergence nulle:
\begin{equation}
div(\overrightarrow{rot}(\overrightarrow{A})) = 0
\label{divrot}
\end{equation}

La divergence peut être vue comme le transposé au signe près, du gradient:
\begin{equation}
div(f\overrightarrow{A}) = f div(\overrightarrow{A}) + \overrightarrow{grad}(f) \cdot \overrightarrow{A}
\end{equation}

- :
\begin{equation}
div(\overrightarrow{A} \land \overrightarrow{B}) = \overrightarrow{B} \cdot \overrightarrow{rot}(\overrightarrow{A}) - \overrightarrow{A} \cdot \overrightarrow{rot}(\overrightarrow{B})
\end{equation}

- :
\begin{equation}
\overrightarrow{rot}(\overrightarrow{rot}(\overrightarrow{A})) = \overrightarrow{grad}(div(\overrightarrow{A})) - \Delta(\overrightarrow{A})
\end{equation}

\subsubsection{Applications}
\paragraph*{Radicaux en carré inverse de la distance}
Le théorème de Gauss nous donne: lorsqu'une loi d'interaction radiale, due à des sources ponctuelles, varie comme le carré inverse de la distance il est possible d'établir que le flux du champ d'interaction à travers une surface fermée est toujours proportionnel à la quantité de sources présentes à l'intérieur de la surface fermée. Par exemple en électromagnétique, avec $Q_{int}$ le nombre de charges dans le volume:
\begin{equation}
\iint_{S}\overrightarrow{E}\cdot\overrightarrow{dS} = \frac{Q_{int}}{\epsilon_{0}}
\end{equation}
Avec le théorème de flux divergence:
\begin{equation}
\iiint_{V}div(\overrightarrow{E})\cdot dV = \frac{Q_{int}}{\epsilon_{0}} = \iiint_{V}\frac{\rho}{\epsilon_{0}}dV
\end{equation}
D'où on en déduit l'équation de Maxwell-Gauss \ref{MaxGauss}. Cette relation est aussi applicable à la gravitation:
\begin{equation}
div(\overrightarrow{G}) = -4\pi G\rho
\end{equation}

\paragraph*{Flux du champ magnétique}
La loi de Maxwell-Thompson \ref{MaxThom} montre que le flux d'un champ magnétique à travers une surface est toujours nul: le champ magnétique est à flux conservatif.
\begin{equation}
\iint_{S}\overrightarrow{B}\cdot \overrightarrow{dS} = \iiint_{V}div(\overrightarrow{B})\cdot dV = 0
\label{consmagn}
\end{equation}


%%%%%%%%%%%%% ROTATIONNEL %%%%%%%%%%%%%
\subsection{Rotationnel \cite{rotationnel}}
\subsubsection{Définition}
L'opérateur rotationnel est un opérateur différentiel aux dérivées partielles qui, à un champ de vecteurs donnés, fait correspondre un autre champ vectoriel.
Il est noté:
\begin{equation}
\overrightarrow{rot}(\overrightarrow{F}) = \overrightarrow{\nabla} \land \overrightarrow{F} = 
\begin{pmatrix}
\frac{\partial F_{z}}{\partial y} - \frac{\partial F_{y}}{z} \\
\frac{\partial F_{x}}{\partial z} - \frac{\partial F_{z}}{x} \\
\frac{\partial F_{y}}{\partial x} - \frac{\partial F_{x}}{y}
\end{pmatrix}
\end{equation}
Le rotationnel exprime la tendance qu'ont les lignes de champs d'un champ vectoriel à tourner autour d'un point: sa circulation locale sur un petit lacet entourant ce point est non nulle quand son rotationnel ne l'est pas. Par exemple:
\begin{description}
\item[Dans une tornade] le vent tourne autour de l'œil du cyclone et le champ vectoriel vitesse du vent a un rotationnel non nul autour de l'œil du cyclone. Plus on se rapproche de l'œil, plus le rotationnel est intense (plus le champ de vorticité/champ tourbillon est intense).
\item[Pour un solide qui tourne a une vitesse $\Omega$] le rotationnel du champ de vitesses et dirigé selon l'axe de rotation, dans le sens direct. La valeur du rotationnel est $2\Omega$.
\end{description}

Le rotationnel d'un champ de vecteurs en un point peut être exprimé comme la circulation locale du champ autour de ce point; cette définition découle du théorème de Green qui donne pour une surface S définie par le contour C:
\begin{equation}
\oint_{C}\overrightarrow{F}\cdot dl = \iint_{S}\overrightarrow{rot}(\overrightarrow{F})\cdot \overrightarrow{dS}
\end{equation}
L'orientation de S et C sont liées tel que $\overrightarrow{dS}\land \overrightarrow{dl}$ est un vecteur dirigé vers la surface (repaire direct).

Le théorème du rotationnel met en relation l'intégrale de volume du rotationnel d'un champ vectoriel avec l'intégrale de surface du même champ. La formule est la suivante, avec S la frontière de V le volume et $\overrightarrow{v}$ le champ vectoriel:
\begin{equation}
\iiint_{V}\overrightarrow{rot}(\overrightarrow{v}) dV = - \iint_{S}\overrightarrow{v}\land \overrightarrow{dS}
\end{equation}

\subsubsection{Propriétés}
Le gradient du rotationel est nul:
\begin{equation}
\overrightarrow{rot}(\overrightarrow{grad}(\overrightarrow{F})) = \overrightarrow{0}
\label{rotgrad}
\end{equation}

La divergence du rotationnel est toujours nulle:
\begin{equation}
div(\overrightarrow{rot}(\overrightarrow{F})) = \overrightarrow{0}
\end{equation}

Le rotationnel du rotationnel est:
\begin{equation}
\overrightarrow{rot}(\overrightarrow{rot}(\overrightarrow{A})) = \overrightarrow{grad}(div(\overrightarrow{A})) - \Delta A
\end{equation}

- :
\begin{equation}
\overrightarrow{rot}(\overrightarrow{A}\cdot\overrightarrow{grad}(\overrightarrow{A})) = \overrightarrow{A}\cdot\overrightarrow{grad}(\overrightarrow{rot}(\overrightarrow{A})) - \overrightarrow{rot}(\overrightarrow{A})\cdot\overrightarrow{grad}(\overrightarrow{A})
\end{equation}


%%%%%%%%%%%%% GRADIENT %%%%%%%%%%%%%
\subsection{Gradient \cite{gradient}}
\subsubsection{Définition}
On définit le gradient comme une grandeur vectorielle qui indique de quelle façon une grandeur physique varie dans l'espace. En mathématiques, le gradient est un vecteur représentant la variation d'une fonction par rapport à ses différents paramètres. En dimension 3:
\begin{equation}
\overrightarrow{grad}(f) = \overrightarrow{\nabla} f = \frac{\partial f}{\partial x}\overrightarrow{x} + \frac{\partial f}{\partial y}\overrightarrow{y} +\frac{\partial f}{\partial z}\overrightarrow{z}
\end{equation}

Construisons le gradient: soit $M$ et $M'$ deux points de l'espace tels que $M$ soit de coordonées $\overrightarrow{u} = (x,y,z)$ et $\overrightarrow{h} = \overrightarrow{MM'} = (h_{x}, h_{y}, h_{z})$. $T$ est une fonction de l'espace. De $M$ à $M'$, $T$ passe de $T(x,y,z)$ à $T(x+h_{x}, y+h_{y}, z+h_{z})$. En première approximation, cette variation est une fonction linéaire de $\overrightarrow{h}$ :
\begin{equation}
T(x+h_{x}, y+h_{y}, z+h_{z}) = T(x,y,z) + \frac{\partial T}{\partial x}(x,y,z)h_{x}  + \frac{\partial T}{\partial y}(x,y,z)h_{y} + \frac{\partial T}{\partial z}(x,y,z)h_{z}
\label{graddl1}
\end{equation}
On créer alors le vecteur appelé gradient de température:
\begin{equation}
\overrightarrow{grad}(T(x,y,z)) = 
\begin{pmatrix}
\frac{\partial T}{\partial x}(x,y,z) & \frac{\partial T}{\partial y}(x,y,z) & \frac{\partial T}{\partial z}(x,y,z)
\end{pmatrix}
\end{equation}
Et on peut écrire la relation \ref{graddl1} sous la forme d'un développement linéaire:
\begin{equation}
T(\overrightarrow{u} + \overrightarrow{h}) = T(\overrightarrow{u}) + \overrightarrow{grad}(T(\overrightarrow{u}))\cdot\overrightarrow{h} + o(\overrightarrow{h})
\end{equation}
Où $o(\overrightarrow{h}$ signifie que le terme qui reste est négligeable devant $\overrightarrow{h}$.

\subsubsection{Propriétés}
EN dimension 2, le gradient normal à une courbe en un point est la droite tangente.

En dimension 3, le gradient normal à une surface en un point est le plan tangent.

Gradient et convexité: Soit une application $f: \Re^{n} \mapsto \Re$ où $n \in \{2,3\}$ par exemple, continuement dérivable. Si l'application $\overrightarrow{grad}(f): \Re^{n} \mapsto \Re^{n}$ est monotomne (resp. strictement monotone) alors $f$ est convexe (resp. strictement convexe). C'est à dire en caractérisant par les cordes:
\begin{equation}
\forall (u,v) \in (\Re^{3})^2, \overrightarrow{grad}_{u}(f)\cdot\overrightarrow{grad}_{v}(f) \geq 0 
\implies
\forall (u,v,\lambda) \in \Re^{3} \times \Re^{3} \times [0,1], f(\lambda u + (1-\lambda)v) \leq \lambda f(u)+(1-\lambda)f(v)
\label{gradconv}
\end{equation}
La propriété \ref{gradconv} est valable même si $f$ n'est pas deux fois dérivable.

D'autres relations vectorielles:
\begin{equation}
\frac{\partial}{\partial t}\overrightarrow{grad}(f) = \overrightarrow{grad}(\frac{\partial f}{\partial t})
\end{equation}
\begin{equation}
div(\overrightarrow{grad}(f)) = \Delta f
\end{equation}
\begin{equation}
\overrightarrow{rot}(\overrightarrow{grad}(f)) = \overrightarrow{0}
\end{equation}




%%%%%%%%%%%%%%%%%%%%%%%%%%%%%%%%%%%%%%%%%%%%%%%%%%%%%%%%%%%%%%%%%%%%%%%%%%%%%%%%%%%
%%%%%%%%%%%%%%%%%%%%%%%%%%%%%%%%%%% ELECTROMAGNETIQUE %%%%%%%%%%%%%%%%%%%%%%%%%%%%%%%%%%%
%%%%%%%%%%%%%%%%%%%%%%%%%%%%%%%%%%%%%%%%%%%%%%%%%%%%%%%%%%%%%%%%%%%%%%%%%%%%%%%%%%%
\section{Électromagnétique}
\subsection{Rappels électromagnétique}

%%%%%%%%%%%%% DENSITE DE COURANT %%%%%%%%%%%%%
\subsubsection{Densité de courant \cite{denscour}}
Elle décrit le courant électrique qui circule à l'échelle locale (en un point du matériau). Il s'agit d'un champ de vecteurs qui associe à tout point de l'espace un vecteur densité de courant. Son unité est : $A.m^{-2}$. Le courant électrique est le débit de charges électriques à travers une surface orienté $\overrightarrow{dS}$.
\begin{equation}
di = \overrightarrow{j}\cdot\overrightarrow{dS} \implies i = \iint_{S}\overrightarrow{j}\cdot\overrightarrow{dS}
\label{densCour}
\end{equation}
Dans la formule \ref{densCour}, le signe de i est lié à l'orientation de $\overrightarrow{dS}$.
Lorsqu'une des dimensions est très petite devant les autres (ex: plaque), on peut alors définir la densité de courant surfacique.

%%%%%%%%%%%%% PERMITTIVITE DIELECTRIQUE %%%%%%%%%%%%%
\subsubsection{Permittivité diélectrique \cite{permelec}}
Elle décrit la réponse d'un milieu à un champ électrique. Au niveau microscopique, elle est liée à la polarisabilité électrique des molécules ou atomes du matériau. Son unité est: Coulombs ou F/m.
La valeur le permittivité diélectrique du vide est:
\begin{equation}
\epsilon_{0} = 8,854187.10^{-2} F\cdot m^{-1}
\end{equation}

On définit la permittivité relative d'un matériau par rapport à celle du vide. Avec $\epsilon_{r}$ permittivité relative(sans unité), $\epsilon$ permittivité du milieu, $\epsilon_{0}$ permittivité du vide:
\begin{equation}
\epsilon_{r} = \frac{\epsilon}{\epsilon_{0}}
\end{equation}
$\epsilon$ est généralement complexe: la partie imaginaire étant liée à l'absorption/émission de champ électromagnétique. La partie réelle et imaginaire sont liées par les relations de Kramers-Kronig.

Dans le très simple cas d'un matériau linéaire, homogène, isotrope, et avec réponse instantanée aux changements du champ électrique, la relation des champs électrique et d’induction à la permittivité est :
\begin{equation}
\overrightarrow{D} = \epsilon \overrightarrow{E}
\label{InducPermElec}
\end{equation}

Dans les milieux plus complexe:
\begin{itemize}
\item Si le matériau n'est pas isotrope, $\epsilon$ est une matrice et $\overrightarrow{D}$ n'est plus colinéaire à $\overrightarrow{E}$.
\item Si le matériau n'est pas homogène, les coefficients $\epsilon_{i,j}$ de la matrice dépendent des coordonnées dans l'espace.
\item Si le matériau n'est pas à réponse instantanée, les coefficients de la matrice dépendent des coordonnées de temps et/ou de fréquence.
\item Si le matériau n'est pas linéaire, la relation \ref{InducPermElec} n'est plus valable.
\end{itemize}


%%%%%%%%%%%%% PERMEABILITE MAGNETIQUE %%%%%%%%%%%%%
\subsubsection{Perméabilité magnétique \cite{permmag}}
En régime linéaire, elle caractérise la faculté d'un matériau à modifier un champ magnétique, i.e modifier les lignes de flux magnétique. Son unité est: $M\cdot m^{-1}$.
La valeur de la perméabilité magnétique du vide ou constante  magnétique est:
\begin{equation}
\mu_{0} = 4\pi 10^{-7} H.m^{-1}
\end{equation}

 Dans le cas d'un régime linéaire, le champ d'excitation magnétique et le champ magnétique sont reliés par la relation:
\begin{equation}
\overrightarrow{B} = \mu\cdot \overrightarrow{H}
\end{equation}

On distingue 3 types de matériaux: dia-magnétiques, paramagnétiques, ferromagnétiques. La perméabilité des matériaux dia-magnétiques et paramagnétiques est très proche de celle du vide. La perméabilité des matériaux ferromagnétiques dépend de l'excitation magnétique $\overrightarrow{H}$. De manière générale, la canalisation du champ magnétique dans un matériaux qui est aussi conducteur est d'autant plus réduite que la fréquence de variation des champs, la perméabilité et la conductivité sont élevées.


\subsection{Équations de Maxwell}
%%%%%%%%%%%%% MAXWELL GAUSS %%%%%%%%%%%%%
\subsubsection{Maxwell-Gauss}
\begin{equation}
div(\overrightarrow{E}) = \frac{\rho}{\epsilon_{0}}
\label{MaxGauss}
\end{equation}
Donne la divergence du champ électrique en fonction de la densité des charges. C'est une équation de "terme source": la densité de charges électriques est une source de champ électrique.
On peut déduire de la formule \ref{MaxGauss}, le champ électrostatique en un point M, créé par une charge ponctuelle q au point O. En notant $\overrightarrow{OM} = r\cdot\overrightarrow{u_{r}}$, on obtient:
\begin{equation}
\overrightarrow{E}(M) = \frac{q}{4\pi\epsilon_{0}r^{2}}\cdot\overrightarrow{u_{r}}
\end{equation}

%%%%%%%%%%%%% MAXWELL THOMSON %%%%%%%%%%%%%
\subsubsection{Maxwell-Thompson}
\begin{equation}
div(\overrightarrow{B}) = 0
\label{MaxThom}
\end{equation}
Cette équation est l'équivalente locale pour le champ magnétique, de l'équation \ref{MaxGauss}, de Maxwell-Gauss, i.e équation avec "terme de source". Le champ magnétique est à flux conservatif, cf \ref{consmagn}. On en déduit aussi qu'il n'existe pas de monopole magnétique: quand on coupe un aimant en deux, on n'obtient pas 1 pôle sud et un pôle nord mais deux nouveaux aimants.

En prenant la réciproque l'équation \ref{divrot}, du divergent, tout champ de vecteurs dont la divergence est nulle peut-être exprimé sous forme de rotationnel. On peut donc définir le potentiel vecteur $\overrightarrow{A}$, tel que:
\begin{equation}
\overrightarrow{B} = \overrightarrow{rot}(\overrightarrow{A})
\end{equation}

%%%%%%%%%%%%% MAXWELL FARADAY %%%%%%%%%%%%%
\subsubsection{Maxwell-Faraday}
\begin{equation}
\overrightarrow{rot}(\overrightarrow{E}) = - \frac{\partial\overrightarrow{B}}{\partial t}
\label{MaxFar}
\end{equation}
Donne la dérivée temporelle du champ magnétique en fonction du champ électrique. Cette équation correspond à un terme variationnel: la variation du champ magnétique crée un champ électrique. L'expression sous sa forme intégrale donne:
\begin{equation}
\oint_{C}\overrightarrow{E}\cdot\overrightarrow{dl} = - \int_{S}\frac{\partial\overrightarrow{B}}{\partial t}\cdot\overrightarrow{dS}
\end{equation}
En prenant la réciproque à l'équation \ref{rotgrad}, du rotationnel,  tout champ de vecteurs dont le rotationnel est nul peut être écrit sous la forme d'un gradient, puis en factorisant la dérivée par rapport à t, on définit un potentiel scalaire électrique V tel que:
\begin{equation}
\overrightarrow{E} = - \overrightarrow{grad}(V) - \frac{\partial\overrightarrow{A}}{\partial t}
\end{equation}
Pour rappel la loi de Faraday qui donne la force électromotrice $\epsilon$ en fonction du flux magnétique $\Phi$ est:
\begin{equation}
\epsilon = - \frac{d\Phi}{dt}
\end{equation}

%%%%%%%%%%%%% MAXWELL AMPERE %%%%%%%%%%%%%
\subsubsection{Maxwell-Ampère}
\begin{equation}
\overrightarrow{rot}(\overrightarrow{B}) = \mu_{0}\overrightarrow{j} + \mu_{0}\epsilon_{0}\frac{\partial\overrightarrow{E}}{\partial t}
\label{MaxAmp}
\end{equation}
Cette formule introduit la densité de courant :$\overrightarrow{j}$.
Sous sa forme intégrale la formule \ref{MaxAmp} lie la circulation sur un contour fermé C et les courants qui traversent une surface S s'appuyant sur ce même contour C:
\begin{equation}
\oint_{C}\overrightarrow{B}\cdot\overrightarrow{dl} = \mu_{0}\iint_{S}\overrightarrow{j}\cdot\overrightarrow{dS} + \mu_{0}\epsilon_{0}\iint_{S}\frac{\partial\overrightarrow{E}}{\partial t}\cdot\overrightarrow{dS}
\end{equation}
En appliquant l'opérateur divergence à la formule \ref{MaxAmp}, on obtient:
\begin{equation}
div(\overrightarrow{j}) + \frac{\partial \rho}{\partial t} = 0
\end{equation}


%%%%%%%%%%%%%%%%%%%%%%%%%%%%%%%%%%%%%%%%%%%%%%%%%%%%%%%%%%%%%%%%%%%%%%%%%%%%%%%%%%%
%%%%%%%%%%%%%%%%%%%%%%%%%%%%%%%%%%%%%% ANTENNES %%%%%%%%%%%%%%%%%%%%%%%%%%%%%%%%%%%%%%
%%%%%%%%%%%%%%%%%%%%%%%%%%%%%%%%%%%%%%%%%%%%%%%%%%%%%%%%%%%%%%%%%%%%%%%%%%%%%%%%%%%
\section{Antennes}
A FAIRE



%%%%%%%%%%%%%%%%%%%%%%%%%%%%%%%%%%%%%%%%%%%%%%%%%%%%%%%%%%%%%%%%%%%%%%%%%%
%%%%%%%%%%%%%%%%%%%%%%%%%%%%%  MECANIQUE DES FLUIDES  %%%%%%%%%%%%%%%%%%%%%%%%%%%%%
%%%%%%%%%%%%%%%%%%%%%%%%%%%%%%%%%%%%%%%%%%%%%%%%%%%%%%%%%%%%%%%%%%%%%%%%%%
\section{Mécanique des fluides}
La mécanique des fluides est une branche de la mécanique des milieux continus qui étudie le comportement des fluides et de leurs forces internes. Elle considère des particules assez petites pour l'analyse mathématique mais plus grande que les molécules pour être décrites comme des fonctions continues, on dit que l'on considère les fluides comme étant continus. Aujourd'hui il y a encore de nombreux problèmes non résolus comme l'étude des turbulences. La recherche utilise systématiquement des outils numériques regroupés sous le terme "Computational fluid dynamics".

\subsection{Propriétés des fluides}
\subsubsection{Présentation du problème \cite{intromecaflu}}
La viscosité d'un fluide est décrite par son nombre de Reynolds\cite{nbreynolds}. Tous les fluides sont visqueux: le mouvement d'une couche de fluide par rapport à une autre est freinée par des phénomènes de frottement. Pour un fluide Newtonien, la force tangentielle est proportionnelle au taux de variation de la vitesse: ce qui conduit aux équations de Navier-Strokes.

Pour un fluide visqueux s'écoulant contre une paroi, la vitesse de la couche de fluide contre la paroi est nulle. Lorsque la viscosité est importante (nombre de Reynolds $<$ 1) l'écoulement est laminaire, c'est l'écoulement de Strokes. En s'éloignant assez des parois les vitesses deviennent quasi-constantes ce qui permet de négliger la viscosité, plus le nombre de Reynolds est élevé dans cette zone plus on peut considérer le fluide comme parfait et ainsi appliquer les équations d'Euler.

Cependant pour dans une première gamme de nombre de Reynolds l'écoulement reste irrotationnel (pas de tourbillons). Pour de plus fortes valeur de nombre de Reynolds, la couche limite engendre un sillage tourbillonnaire (avec rotationnel). Pour des valeurs encore plus élevée du nombre de Reynolds, la couche limite est laminaire en amont mais turbulente en aval.

Il faut aussi considérer la compressibilité du fluide, décrite par le nombre de Mach\cite{nbmach}.

Dans un fluide on observe différents régimes d'écoulement. 

Mathématiquement on distingue les régimes:
\begin{description}
\item[permanents (ou stationnaires)] les grandeurs ne dépendent pas du temps.
\item[uniformes] la vitesse ne dépend pas du point considéré.
\end{description}

Physiquement on distingue les régimes:
\begin{description}
\item[laminaires] les couches de fluides glissent les unes par rapport aux autres, les vitesses sont continues.
\item[turbulents] les vitesses sont discontinues, les couches de fluides s'interpénètrent de manière aléatoire.
\item[tourbillonnaires] qui apparait fréquemment dans la transition laminaire-turbulent.
\end{description}

\subsubsection{Fluide compressible et incompressible}
Si les changement de densité d'un fluide ont des effets significatif sur les résultats, on dit que le fluide est compressible. Dans le cas contraire le fluide est incompressible et les changements de densité sont ignorés.

Le nombre de Mach\cite{nbmach} permet de déterminer la compressibilité d'un fluide. On considère un fluide comme compressible pour des nombre de Mach supérieurs à $0.3$.

Explication de la compressibilité: en acoustique, l'air doit être considéré comme un fluide compressible sinon la propagation du son dans l'air serait instantanée (vitesse infinie). Pour note, la vitesse de propagation du son dans un fluide de compressibilité $\chi$ est:
\begin{equation}
c^{2} = (\rho_{0}\chi)^{-1}
\end{equation}

\subsubsection{Viscosité}
Dans les problèmes où les frottements entre les couches de fluides ont des effets non négligeables sur la solution, les fluides sont considérés visqueux. Sinon ils sont appelés non-visqueux. 

Le nombre de Reynolds\cite{nbreynolds} est employé pour estimer quel type d'équations doit être utiliser pour résoudre un problème: pour des écoulements à faible nombre de Reynolds (près d'une paroi) l'écoulement est laminaire, on utilisera les équations de Navier-Strokes. Pour des nombres de Reynolds élevés, l'écoulement pourra être considéré comme non-visqueux (donc non laminaire) et les équations d'Euler seront utilisée.

ATTENTION: il arrive que même si le nombre de Reynolds est élevé, il faille considérer les effets de viscosité, par exemple pour un écoulement autour d'une aile d'avion (cf paradoxe de D'Alembert\cite{paradoxealembert}).

\subsubsection{Écoulement stationnaire et instationnaire}
Le fluide est considéré comme stationnaire si toutes ses propriétés sont considérées comme constantes avec le temps. Ceci constitue une bonne approximation pour des problèmes tels que la trainée d'une aile, la poussée ou un fluide traversant un tuyau. Dans ce cas les équations de Navier-Strokes peuvent être simplifiées.

Si un fluide est incompressible, non-visqueux et stationnaire il peut être résolu avec l'écoulement potentiel découlant de l'équation de Laplace.

\subsubsection{Écoulement laminaire et turbulences}
La turbulence est un écoulement avec remous et un aspect aléatoire apparent. S'il n'y a pas de turbulences, l'écoulement est laminaire. Les turbulences obéissent à l'équation de Navier-Strokes, cependant la résolution est si complexe qu'il n'est pas possible actuellement de les résoudre numériquement avec les principes de base. Les turbulences sont plutôt modélisées avec un des nombreux modèles de turbulence couplée avec un résolveur de flux laminaires.


\bibliographystyle{plain}
\bibliography{biblio}


\end{document} %%% END DOC



























































































